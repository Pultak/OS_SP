% !TeX spellcheck = cs_CZ
\documentclass[ 12pt, a4paper]{article}
\usepackage{lmodern}
\usepackage[utf8]{inputenc}
\usepackage[czech]{babel}
\usepackage{hyperref}
\usepackage{graphicx}
\usepackage{caption}
\usepackage{indentfirst}
\usepackage{listings}
\usepackage{color}
\usepackage{amsmath}
\usepackage{array}
\usepackage{amsmath} 
\usepackage{graphicx}
\usepackage{listings}
\usepackage{xcolor}

\newenvironment{conditions}
{\par\vspace{\abovedisplayskip}\noindent\begin{tabular}{>{$}l<{$} @{${}={}$} l}}
	{\end{tabular}\par\vspace{\belowdisplayskip}}
\definecolor{dkgreen}{rgb}{0,0.6,0}
\definecolor{gray}{rgb}{0.5,0.5,0.5}
\definecolor{mauve}{rgb}{0.58,0,0.82}


\colorlet{punct}{red!60!black}
\definecolor{background}{HTML}{EEEEEE}
\definecolor{delim}{RGB}{20,105,176}
\colorlet{numb}{magenta!60!black}

\lstdefinelanguage{json}{
	basicstyle=\normalfont\ttfamily,
	numbers=left,
	numberstyle=\scriptsize,
	stepnumber=1,
	numbersep=8pt,
	showstringspaces=false,
	breaklines=true,
	frame=lines,
	backgroundcolor=\color{background},
	literate=
	{:}{{{\color{punct}{:}}}}{1}
	{,}{{{\color{punct}{,}}}}{1}
	{\{}{{{\color{delim}{\{}}}}{1}
	{\}}{{{\color{delim}{\}}}}}{1}
	{[}{{{\color{delim}{[}}}}{1}
	{]}{{{\color{delim}{]}}}}{1},
}

\lstset{frame=tb,
	language=json,
	aboveskip=3mm,
	belowskip=3mm,
	showstringspaces=false,
	columns=flexible,
	basicstyle={\small\ttfamily},
	numbers=none,
	numberstyle=\tiny\color{gray},
	keywordstyle=\color{blue},
	commentstyle=\color{dkgreen},
	stringstyle=\color{mauve},
	breaklines=true,
	breakatwhitespace=true,
	tabsize=3
}


\begin{document}
\def\code#1{\texttt{#1}}

\pagenumbering{gobble} %bez cisel
%
%Titulni strana
\centerline{\includegraphics[width=12cm]{logo.png}}
\vspace*{50px}
\begin{center}
	{\LARGE\bf\noindent KIV/OS \\ Simulace operačního systému}\\
	\vspace*{40px}  
	
	Tomáš Ott\\
	\texttt{ottt@students.zcu.cz}\\ 
	(A21N0062P)\\
	\vspace*{10px}  
	
	Petr Kocián\\
	\texttt{ottt@students.zcu.cz}\\ 
	(A21N0032P)\\
	\vspace*{10px}  
	
	Matěj Zeman\\
	\texttt{zemanm98@students.zcu.cz}\\ 
	(A21N0080P)\\
	
	\vspace*{\fill}  
	\hspace*{\fill} \today \\
\end{center}
\newpage
%Obsah
\tableofcontents
\newpage

\pagenumbering{arabic}


%%%%%%%%%%%%%%%%%%%%%%%%%%%%%%%%%%%%%
%%%%%%%%%%%%%%%%%%%%%%%%%%%%%%%%%%%%%
%%%%%%%%%%%%%%%%%%%%%%%%%%%%%%%%%%%%%
\section{Zadání}

\begin{itemize}
	\item Vytvořte virtuální stroj, který bude simulovat OS
	\item Součástí bude shell s gramatikou cmd, tj. včetně exit
	\item Vytvoříte ekvivalenty standardních příkazů a programů
	
	\begin{itemize}
		\item echo, cd, dir, md, rd, type, find /v "" /c , sort, tasklist, shutdown
		\begin{itemize}
			\item cd a dir musí umět relativní cesty
			\item dir musí umět /S
			\item echo musí umět @echo on a off
			\item type musí umět vypsat jak stdin, tak musí umět vypsat soubor
		\end{itemize}
		\item Dále vytvoříte programy rgen a freq
		\item rgen bude vypisovat náhodně vygenerovaná čísla v plovoucí čárce na stdout, dokud mu nepřijde EOF, ETX, nebo EOT
		\item freq bude číst z stdin a sestaví frekvenční tabulku bytů, kterou pak vypíše pro všechny byty s frekvencí větší než 0 ve formátu: “0x\%hhx : \%d”
	\end{itemize}
	\item Implementujte roury a přesměrování
	\item Nebudete přistupovat na souborový systém, ale použijete simulovaný disk se souborovým systémem FAT12
	\begin{itemize}
		\item Součástí zadání je obraz diskety FreeDOS 1.2, vůči kterému budou prováděny testy při hodnocení semestrální práce
		\item V DOSBoxu ho lze připojit příkazem: imgmount d fdos1\_2\_floppy.img -size 512,63,16,1
	\end{itemize}
\end{itemize}


Při zpracování tohoto zadání použijte a dále pracujte s kostrou tohoto řešení, kterou najdete v archívu os\_simulator.zip. Součástí archívu, ve složce compiled, je soubory checker.exe a test.exe. Soubor checker.exe je validátor semestrálních prací. Soubor test.exe generuje možný testovací vstup pro vaši semestrální práci.

Vaše vypracování si před odevzdáním zkontrolujte programem checker.exe. V souboru checker.ini si upravte položku Setup\_Environment\_Command, v sekci General, tak, aby obsahovala cestu dle vaší instalace Visual Studia. Např. vzorové odevzdání otestujete příkazem ''compiled\\vzorove odevzdani.zip", spuštěného v kořenovém adresáři rozbaleného archívu. Odevzdávaný archív nemá obsahovat žádné soubory navíc a program musí úspěšně proběhnout.


%%%%%%%%%%%%%%%%%%%%%%%%%%%%%%%%%%%%%
%%%%%%%%%%%%%%%%%%%%%%%%%%%%%%%%%%%%%
%%%%%%%%%%%%%%%%%%%%%%%%%%%%%%%%%%%%%
\newpage
\section{Analýza problému}\label{struktura}


%%%%%%%%%%%%%%%%%%%%%%%%%%%%%%%%%%%%%
%%%%%%%%%%%%%%%%%%%%%%%%%%%%%%%%%%%%%
%%%%%%%%%%%%%%%%%%%%%%%%%%%%%%%%%%%%%
\newpage
\section{Popis implementace}\label{prikazy}

%%%%%%%%%%%%%%%%%%%%%%%%%%%%%%%%%%%%%
%%%%%%%%%%%%%%%%%%%%%%%%%%%%%%%%%%%%%
%%%%%%%%%%%%%%%%%%%%%%%%%%%%%%%%%%%%%
\newpage
\section{Závěr} \label{diskuze}

\end{document}